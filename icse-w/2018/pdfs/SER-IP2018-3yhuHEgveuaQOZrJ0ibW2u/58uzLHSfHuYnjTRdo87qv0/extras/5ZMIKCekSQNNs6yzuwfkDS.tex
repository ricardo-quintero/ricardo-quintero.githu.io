
\title[Continuously evaluated research projects]{Continuously evaluated research projects in \\ collaborative decoupled environments}
%\titlenote{Produces the permission block, and
%	copyright information}
%\subtitle{Extended Abstract}
%\subtitlenote{The full version of the author's guide is available as
%	\texttt{acmart.pdf} document}

\author{Oliver Schmidts}
\email{schmidts@fh-aachen.de}
\affiliation{%
	\institution{FH Aachen - University of Applied Sciences 
		\\Medical Engineering and Technomathematics\\}
	\streetaddress{Heinrich-Mussmann-Straße 1}
	\city{Jülich} 
	\state{Germany} 
	\postcode{52428}
}

\author{Bodo Kraft}
\email{kraft@fh-aachen.de}
\affiliation{%
	\institution{FH Aachen - University of Applied Sciences 
		\\Medical Engineering and Technomathematics\\}
	\streetaddress{Heinrich-Mussmann-Straße 1}
	\city{Jülich} 
	\state{Germany} 
	\postcode{52428}
}

\author{Marc Schreiber}
\email{marc.schreiber@fh-aachen.de}
\affiliation{%
	\institution{FH Aachen - University of Applied Sciences 
		\\Medical Engineering and Technomathematics\\}
	\streetaddress{Heinrich-Mussmann-Straße 1}
	\city{Jülich} 
	\state{Germany} 
	\postcode{52428}
}


\author{Albert Zündorf}
\email{zuendorf@uni-kassel.de}
%\authornote{Dr.~Trovato insisted his name be first.}
%\orcid{1234-5678-9012}
\affiliation{%
	\institution{University of Kassel\\
		Software Engineering Research Group\\
		}
	\city{Kassel} 
	\state{Germany}
}


\renewcommand{\shortauthors}{O. Schmidts et al.}

\begin{abstract}
Often, research results from collaboration projects are not transferred into productive environments even though approaches are proven to work in demonstration prototypes. These demonstration prototypes are usually too fragile and error-prone to be transferred easily into productive environments. A lot of additional work is required.

Inspired by the idea of an incremental delivery process, we introduce an architecture pattern, which combines the approach of \acrlong{MEDIATION} with microservices for the ease of integration. It enables keeping track of project goals over the course of the collaboration while every party may focus on their expert skills: researchers may focus on complex algorithms, practitioners may focus on their business goals.

Through the simplified integration (intermediate) research results can be introduced into a productive environment which enables getting an early user feedback and allows for the early evaluation of different approaches. The practitioners' business model benefits throughout the full project duration.





%Research collaborations provide opportunities for both practitioners and researchers: practitioners need solutions for difficult business challenges and researchers are looking for hard problems to solve and publish. Nevertheless, research collaborations carry the risk that practitioners focus on quick solutions too much and that researchers tackle theoretical problems, resulting in products which do not fulfill the project requirements. 

%In this paper we introduce an approach extending the ideas of agile and lean software development. It helps practitioners and researchers keep track of their common research collaboration goal: a scientifically enriched software product which fulfills the needs of the practitioner's business model. 

%This approach gives first-class status to application-oriented metrics that measure progress and success of a research collaboration continuously. Those metrics are derived from the collaboration requirements and help to focus on a commonly defined goal.\cite{schreiber_metrics_2017}

%An appropriate tool set evaluates and visualizes those metrics with minimal effort, and all participants will be pushed to focus on their tasks with appropriate effort. Thus project status, challenges and progress are transparent to all research collaboration members at any time. 
\end{abstract}

\begin{CCSXML}
	<ccs2012>
	<concept>
	<concept_id>10011007.10011074.10011134.10011135</concept_id>
	<concept_desc>Software and its engineering~Programming teams</concept_desc>
	<concept_significance>500</concept_significance>
	</concept>
	<concept>
	<concept_id>10011007.10011074.10011092.10010876</concept_id>
	<concept_desc>Software and its engineering~Software prototyping</concept_desc>
	<concept_significance>300</concept_significance>
	</concept>
	<concept>
	<concept_id>10011007.10011074.10011099.10011693</concept_id>
	<concept_desc>Software and its engineering~Empirical software validation</concept_desc>
	<concept_significance>300</concept_significance>
	</concept>
	</ccs2012>
\end{CCSXML}

\ccsdesc[500]{Software and its engineering~Programming teams}
\ccsdesc[300]{Software and its engineering~Software prototyping}
\ccsdesc[300]{Software and its engineering~Empirical software validation}

\keywords{Research Best Practices, Research Collaboration Management, Metrics, Lean Software Development, Software Architecture}

\maketitle

